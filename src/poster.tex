% Gemini theme
% https://github.com/anishathalye/gemini

\documentclass[final]{beamer}

% ====================
% Packages
% ====================

\usepackage[T1]{fontenc}
\usepackage{lmodern}
\usepackage[size=custom,width=76.2,height=152.4,scale=1.0]{beamerposter}
\usetheme{gemini}
\usecolortheme{asteroidcity3}
\usepackage{graphicx}
\usepackage{booktabs}
\usepackage{tikz}
\usepackage{pgfplots}
\pgfplotsset{compat=1.14}
\usepackage{anyfontsize}
\usepackage{multicol}
\usepackage[numbers]{natbib}

% ====================
% Lengths
% ====================

% If you have N columns, choose \sepwidth and \colwidth such that
% (N+1)*\sepwidth + N*\colwidth = \paperwidth
\newlength{\sepwidth}
\newlength{\colwidth}
\setlength{\sepwidth}{0.0167\paperwidth}
\setlength{\colwidth}{0.45\paperwidth}
\newcommand{\separatorcolumn}{\begin{column}{\sepwidth}\end{column}}

% ====================
% Title
% ====================
\title{Observational Constraints on Models of Energy Release in Long-lived Active Region Loops}
\author{
  W. T. Barnes \inst{1}\textsuperscript{,}\inst{2} \and
  H. P. Warren \inst{3} \and
  J. W. Reep \inst{3}
}
\institute[]{
  \inst{1} Department of Physics, American University \samelineand
  \inst{2} Heliophysics Science Division, NASA Goddard Space Flight Center \and
  \inst{3} Space Science Division, Naval Research Laboratory
}

% ====================
% Footer (optional)
% ====================
\footercontent{
  \href{https://github.com/wtbarnes/hinode-2023-loops-poster}{github.com/wtbarnes/hinode-2023-loops-poster} \hfill
  Hinode-16/IRIS-13 --- Niigata, Japan --- 25--29 September 2023 \hfill
  \href{mailto:wbarnes@american.edu}{wbarnes@american.edu}
}

% ====================
% Logo (optional)
% ====================
\logoright{\includegraphics[height=8cm]{static/AU_stacked_logo.png}}
\logoleft{\includegraphics[height=10cm]{static/sunpy_logo_portrait_powered.png}}

% ====================
% Body
% ====================

\begin{document}

\begin{frame}[t]
\begin{columns}[t]
\separatorcolumn

\begin{column}{\colwidth}

  \begin{block}{Introduction}

    \citet{lenz_temperature_1999,patsourakos_inability_2004}

  \end{block}

  \begin{block}{NOAA AR 1575 Oberved by AIA, EIS, and XRT}

  \end{block}

  \begin{alertblock}{Isolating Loop Structures}

  \end{alertblock}

  \begin{block}{Density Diagnostics}

  \end{block}

  \begin{block}{Filter Ratio}

  \end{block}



\begin{block}{Differential Emission Measure}

\end{block}

\begin{block}{Time Lag Analysis}
\end{block}

\end{column}

\separatorcolumn

\begin{column}{\colwidth}

  \begin{block}{Constraining Loop Geometries}
    
  \end{block}

  \begin{block}{Hydrodynamic Modeling}
  \end{block}

  \begin{block}{Synthetic Observables}

  \end{block}

  \begin{alertblock}{Assessing Heating Models}

    \begin{table}
      \centering
      \begin{tabular}{l r r c}
        \toprule
        \textbf{First column} & \textbf{Second column} & \textbf{Third column} & \textbf{Fourth} \\
        \midrule
        Foo & 13.37 & 384,394 & $\alpha$ \\
        Bar & 2.17 & 1,392 & $\beta$ \\
        Baz & 3.14 & 83,742 & $\delta$ \\
        Qux & 7.59 & 974 & $\gamma$ \\
        \bottomrule
      \end{tabular}
      \caption{A table caption.}
    \end{table}

  \end{alertblock}

  \begin{block}{Conclusions}
    
  \end{block}

  \begin{block}{References}
    \scriptsize
    \begin{multicols}{2}
      \bibliographystyle{aasjournal.bst}
      \bibliography{references.bib}
    \end{multicols}
  \end{block}

\end{column}

\separatorcolumn
\end{columns}
\end{frame}

\end{document}
