% Gemini theme
% https://github.com/anishathalye/gemini

\documentclass[final]{beamer}

% ====================
% Packages
% ====================

\usepackage[T1]{fontenc}
\usepackage{lmodern}
\usepackage[size=custom,width=76.2,height=167.64,scale=1.0]{beamerposter}
\usetheme{gemini}
\usecolortheme{asteroidcity3}
\usepackage{graphicx}
\usepackage{booktabs}
\usepackage{tikz}
\usepackage{pgfplots}
\pgfplotsset{compat=1.14}
\usepackage{anyfontsize}
\usepackage{multicol}
\usepackage[numbers]{natbib}
\usepackage{import}
\usepackage{siunitx}
\usepackage{float}

% ====================
% Lengths
% ====================

% If you have N columns, choose \sepwidth and \colwidth such that
% (N+1)*\sepwidth + N*\colwidth = \paperwidth
\newlength{\sepwidth}
\newlength{\colwidth}
\setlength{\sepwidth}{0.0167\paperwidth}
\setlength{\colwidth}{0.45\paperwidth}
\newcommand{\separatorcolumn}{\begin{column}{\sepwidth}\end{column}}
\setlength\abovecaptionskip{-3pt}

% ====================
% Title
% ====================
\title{Observational Constraints on Models of Energy Release in Long-lived Active Region Loops}
\author{
  W. T. Barnes \inst{1}\textsuperscript{,}\inst{2} \and
  H. P. Warren \inst{3} \and
  J. W. Reep \inst{3}
}
\institute[]{
  \inst{1} Department of Physics, American University \samelineand
  \inst{2} Heliophysics Science Division, NASA Goddard Space Flight Center \and
  \inst{3} Space Science Division, Naval Research Laboratory
}

% ====================
% Footer (optional)
% ====================
\footercontent{
  \href{https://github.com/wtbarnes/hinode-2023-loops-poster}{github.com/wtbarnes/hinode-2023-loops-poster} \hfill
  Hinode-16/IRIS-13 --- Niigata, Japan --- 25--29 September 2023 \hfill
  \href{mailto:wbarnes@american.edu}{wbarnes@american.edu}
}

% ====================
% Logo (optional)
% ====================
\logoright{\includegraphics[height=8cm]{static/AU_stacked_logo.png}}
\logoleft{\includegraphics[height=10cm]{static/sunpy_logo_portrait_powered.png}}

% ====================
% Body
% ====================

\begin{document}

\begin{frame}[t]
\begin{columns}[t]
\separatorcolumn

\begin{column}{\colwidth}

  \begin{block}{Introduction}

    \begin{itemize}
      \item Long "warm" EUV loops on periphery of active regions (ARs) observed to:
      \begin{itemize}
        \item Be \alert{approximately steady}, lasting much longer than a cooling time
        \item Have \alert{narrow, nearly isothermal emission measure distributions centered near \SI{1}{\mega\kelvin}} \citep{aschwanden_three-dimensional_1999,aschwanden_evidence_2000}
        \item Be \alert{overdense} compared to densities expected from hydrostatic scaling laws \citep{winebarger_transition_2003}
        \item Have \alert{flat temperature structures} as determined by the \SI{195}{\angstrom}/\SI{171}{\angstrom} filter ratio \citep{aschwanden_three-dimensional_1999,lenz_temperature_1999}
      \end{itemize}
      \item \textbf{\alert{Problem:}} hydrostatic models fail to produce sufficiently high densities while impulsive heating scenarios lead to overly broad temperature distributions
      \item \textbf{\alert{Goal:}} Constrain observational properties of long warm loops spatially, temporally and over a broad range of temperatures and wavelengths
      \item \textbf{\alert{Goal:}} Constrain possible time-independent heating scenarios by comparing forward-modeled observables from field-aligned hydrodynamic models to observational diagnostics
    \end{itemize}

  \end{block}

  \begin{block}{NOAA AR 1575 Oberved by AIA, EIS, and XRT}

    \begin{columns}[c]
      \begin{column}{0.35\colwidth}
        \begin{itemize}
          \item EIS raster started 2012-09-24 10:50:26 and ended 2012-09-24 13:56:09
          \item XRT synoptic data available for 2012-09-24 10:03
          \item AIA data selected between 2012-09-24 09:00:00 and 2012-09-24 21:00:00
          \item Fit 38 lines in each pixel of EIS raster to derive intensity and velocity using \texttt{eispac} \citep{weberg_eispac_2023}
          \item Align AIA EUV images to EIS FOV, normalizing by exposure time and correcting for degradation using \texttt{aiapy} \citep{barnes_aiapy_2020} 
        \end{itemize}
      \end{column}
      \begin{column}{0.65\colwidth}
        \begin{figure}[H]
          \centering
          \import{figures/}{ar_maps.pgf}
          \caption{NOAA AR 1575 as observed on 2012-09-24 by SDO/AIA (top row), \textit{Hinode}/EIS (bottom left) and \textit{Hinode}/XRT (bottom right). The red dots and black lines denote the manually-identified loop.}
          \label{fig:ar_maps}
        \end{figure}
      \end{column}
    \end{columns}

  \end{block}

  \begin{block}{Isolating Loop Structures}
    \begin{itemize}
      \item Manually identify points in time-averaged MGN-filtered \SI{171}{\angstrom} images that fall within EIS FOV and persist for $\approx\SI{12}{\hour}$.
      \item Create 2D pixel mask from each image assuming a loop width of \SI{45}{\arcsecond} with dimensions corresponding to the loop-aligned ($s_\parallel$) and cross-loop ($s_\perp$) directions.
      \item To further isolate loop structure, compute linear background model from emission at $s_\perp=\SI{0}{\arcsecond},\SI{45}{\arcsecond}$
    \end{itemize}
    \vspace{-50pt}
    \begin{figure}[H]
      \centering
      \import{figures/}{straightened_loops.pgf}
      \caption{Background-subtracted ``straightened'' loops as a function of $s_\parallel$ and $s_\perp$ as extracted from each image in \autoref{fig:ar_maps}. pixels are masked white if they fall below the background.}
      \label{fig:straightened_loops}
    \end{figure}

  \end{block}
  \vspace{-30px}
  \begin{block}{Density Diagnostics}

    \begin{itemize}
      \item Use method of \citet{young_high-precision_2009} to derive density from Fe XII \SI{186.880/195.119}{\angstrom} and Fe XIII \SI{203.826/202.044}{\angstrom} line pairs as observed by EIS
      \item Density structure approximately flat as a function of $s_\parallel$, both density distributions peaked at $\approx\SI{1e9}{\per\cm\cubed}$
    \end{itemize}
    \vspace{-50pt}
    \begin{figure}
      \centering
      \import{figures/}{density_diagnostics.pgf}
      \caption{Derived density from the Fe XII (top row) and Fe XIII (second row) line ratios. \textbf{Bottom left:} panel shows the theoretical line ratio as a function of density for the Fe XII and XIII line pairs. \textbf{Bottom right:} Distribution of densities from the maps in the first two rows.}
      \label{fig:density_diagnostic}   
    \end{figure}

  \end{block}
  \vspace{-30pt}
  \begin{block}{Temperature Diagnostics}

    \heading{Filter Ratio}
    \vspace{-50pt}
    \begin{figure}[H]
      \centering
      \import{figures/}{temperature_diagnostics.pgf}
      \caption{Temperature as derived from the ratio of the \SI{193}{\angstrom} and \SI{171}{\angstrom} filters. \textbf{Top:} Time-averaged temperature as a function of $s_\parallel,s_\perp$. \textbf{Middle:} Temperature averaged over $s_\perp$ as a function of the \SI{12}{\hour} AIA observing window. \textbf{Bottom:} Time- and $s_\perp$-averaged temperature. The shaded region denotes $1\sigma$ of the distribution of temperatures in time and $s_\perp$.}
      \label{fig:filter_ratio}
    \end{figure}
    \vspace{-30pt}
    \heading{Differential Emission Measure Inversion}
    \vspace{-50pt}
    \begin{figure}[H]
      \centering
      \import{figures/}{dem.pgf}
      \caption{\textbf{Left:} Contribution functions for all observed EIS lines and temperature response functions for several filters from AIA and XRT. \textbf{Right:} EM \textit{loci} curves and inverted emission measure distribution (black) computed using the MCMC method of \citet{kashyap_markov-chain_1998}. The intensities are averaged in $s_\parallel$ and the value of $s_\perp$ is chosen where the Fe XII \SI{195.119}{\angstrom} intensity is maximized.}
      \label{fig:dem_observed}
    \end{figure}

  \end{block}

\end{column}

\separatorcolumn

\begin{column}{\colwidth}

  \begin{block}{Time Lag Analysis}

    \begin{itemize}
      \item Apply time-lag analysis method of \citet{viall_evidence_2012,barnes_understanding_2019} to understand cooling patterns along loop.
      \item Spatially coherent positive time lags in the 193-171 and 211-193 indicate cooling between around \SI{2}{\mega\K} to just below \SI{1}{\mega\K}
      \item Incoherent time lags in 335-211 indicate little coherent evolution above \SI{2}{\mega\K}
      \item Zero time lags in 171-131 indicate no cooling below \SI{0.5}{\mega\K}
    \end{itemize}

    \begin{figure}
      \centering
      \import{figures/}{time_lags.pgf}
      \caption{Time lag analysis of the non-background-subtracted traced loops for four different AIA channel pairs. The value of each pixel denotes the time lag, in seconds, that maximizes the cross-correlation between the two channels.}
      \label{fig:timelags}
    \end{figure}
  
  \end{block}

  \begin{block}{Constraining Loop Geometries}

    \begin{columns}[b]
      \begin{column}{0.5\colwidth}
        \centering
        \begin{itemize}
          \item Compute PFSS solution from HMI synoptic magnetogram
          \item Select fieldlines coincident with EUV loops to constrain $B(s)$, $g_\odot(s)$, and area expansion ($1/B(s)$)
        \end{itemize}
        \begin{figure}
          \centering
          \includegraphics[width=0.5\colwidth]{figures/tmp_modeling/aia-171-with-loops.pdf}
          \caption{AIA \SI{171}{\angstrom} with field lines from a PFSS model overlaid that most closely intersect with the manually traced loop.}
          \label{fig:aia_fieldlines_overlaid}
        \end{figure}
      \end{column}
      \begin{column}{0.5\colwidth}
        \begin{figure}
          \centering
          \includegraphics[width=0.5\colwidth]{figures/tmp_modeling/bundle_b_field_expansion.pdf}
          \caption{\textbf{Top:} Magnetic field strength as a function of $s$ for each model fieldline. The solid line denotes the average over all fieldlines. \textbf{Bottom}: Area expansion derived from the inverse of the average magnetic field strength.}
          \label{fig:b_field_bundle}
        \end{figure}
      \end{column}
    \end{columns}
    
  \end{block}

  \begin{block}{Hydrodynamic Modeling}
    \begin{columns}[c]
      \begin{column}{0.4\colwidth}
        \begin{itemize}
          \item Calculate thermodynamic structure as a function of $s$ using the HYDRAD model \citep{bradshaw_self-consistent_2003,bradshaw_influence_2013}
          \item Volumetric heating can be parameterized as a function of location $s_0$ and scale height $s_H$, \begin{equation*}
            E = \sum_i E_{0,i} \exp \left[-\frac{(s-s_{i,0})^2}{2s_H^2}\right]
          \end{equation*}
          \item Motivated by steady appearance of loops, the energy deposition is steady in all cases
        \end{itemize}
      \end{column}
      \begin{column}{0.6\colwidth}
        \begin{table}
          \begin{tabular}{ccccc}
            \toprule
            Run & $s_H$ & $E_{0,L}$ & $E_{0,R}$ & Expansion? \\
                & Mm & erg cm$^{-3}$ s$^{-1}$ & erg cm$^{-3}$ s$^{-1}$ &  \\
            \midrule
            1 & $\infty$ & 8.7e-06 & 8.7e-06 & No \\
            2 & $\infty$ & 8.7e-06 & 8.7e-06 & Yes \\
            3 & 3.32 & 0.002 & 0.002 & Yes \\
            4 & 3.32 & 0 & 0.004 & No \\
            5 & 3.32 & 0 & 0.004 & Yes \\
            \bottomrule
          \end{tabular}
        \end{table}          
      \end{column}
    \end{columns}
  \end{block}

  \begin{block}{Synthetic Observables}
    \heading{Temperature and Density Structure}

    \begin{columns}[c]
      \begin{column}{0.5\colwidth}
        \begin{figure}
          \centering
          \includegraphics[width=0.5\colwidth]{figures/tmp_modeling/model-density-comparison.pdf}
          \caption{Temperature (top) and density (bottom) as a function of field-aligned coordinate for each heating scenario.}
          \label{fig:temperature_density_profiles}
        \end{figure}
      \end{column}
      \begin{column}{0.5\colwidth}
        \begin{itemize}
          \item In all runs except Run 3, simulation reaches steady state and take final timestamp. 
          \item In Run 3, all simulation to run until the condensation formation cycle is periodic due to TNE
          \item TNE (Run 3) leads to the flattest temperature profile
          \item Including field-aligned expansion (Runs 2,3,5) increases density across the loop structure
        \end{itemize}
      \end{column}
    \end{columns}

    \heading{Emission Measure Distribution}

    \begin{columns}[c]
      \begin{column}{0.5\colwidth}
        \begin{itemize}
          \item $\mathrm{EM}(T)$ distributions derived by binning the temperatures $\pm\SI{100}{\mega\m}$ of apex, and weighted by $n^2$. \item In Run 3, average over 1 TNE cycle period to capture the temperature evolution of the condensation formation.
          \item All runs show relatively isothermal $\mathrm{EM}(T)$ except Run 3
          \item TNE leads to broad $\mathrm{EM}(T)$ but no emission above the peak of the $\mathrm{EM}(T)$
        \end{itemize}
      \end{column}
      \begin{column}{0.5\colwidth}
        \begin{figure}
          \includegraphics[width=0.5\colwidth]{figures/tmp_modeling/model-dem-comparison.pdf}
          \label{fig:model_dem}
          \caption{Emission measure distribution for each heating scenario.}
        \end{figure}
      \end{column}
    \end{columns}

  \end{block}

  \begin{block}{Assessing Heating Models}

    \begin{columns}[c]
      \begin{column}{0.4\colwidth}
        \begin{table}
          \centering
          \begin{tabular}{c c c c}
            \toprule
            \textbf{Run} & \textbf{Overdense?} & \textbf{Isothermal?} & \textbf{Flat?} \\
            \midrule
            1 & No & Yes &  Kinda? \\
            2 & No & Yes & Kinda? \\
            3 & No & No &  Yes \\
            4 & Kinda? & Yes &  No \\
            5 & No & Yes & No \\
            \bottomrule
          \end{tabular}
        \end{table}    
      \end{column}
      \begin{column}{0.59\colwidth}
        \begin{itemize}
          \item All models show densities below \SI{1e9}{\per\cubic\cm} by $\ge2$
          \item Run 4 (siphon flow with expansion) most promising, but velocities too high
          \item Time-dependent solutions of $L\sim\SI{300}{\mega\m}$ that are sufficiently overdense are too broad in $T$ and hot
        \end{itemize}
      \end{column}
    \end{columns}

  \end{block}

  \begin{block}{References}
    \scriptsize
    WTB was supported by NASA's \textit{Hinode} program.
    \textit{Hinode} is a Japanese mission developed and launched by ISAS/JAXA with NAOJ as a domestic partner and NASA and STFC (UK) as international partners.
    It is operated by these agencies in cooperation with ESA and NSC (Norway).
    \begin{multicols}{2}
      \bibliographystyle{aasjournal.bst}
      \bibliography{references.bib}
    \end{multicols}
  \end{block}

\end{column}

\separatorcolumn
\end{columns}
\end{frame}

\end{document}
