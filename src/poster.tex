% Gemini theme
% https://github.com/anishathalye/gemini

\documentclass[final]{beamer}

% ====================
% Packages
% ====================

\usepackage[T1]{fontenc}
\usepackage{lmodern}
\usepackage[size=custom,width=84.1,height=167.64,orientation=portrait,scale=1.0]{beamerposter}
\usetheme{gemini}
\usecolortheme{asteroidcity3}
\usepackage{graphicx}
\usepackage{booktabs}
\usepackage{tikz}
\usepackage{pgfplots}
\pgfplotsset{compat=1.14}
\usepackage{anyfontsize}
\usepackage{multicol}
\usepackage[numbers]{natbib}
\usepackage{import}
\usepackage{siunitx}
\usepackage{float}

% ====================
% Lengths
% ====================

% If you have N columns, choose \sepwidth and \colwidth such that
% (N+1)*\sepwidth + N*\colwidth = \paperwidth
\newlength{\sepwidth}
\newlength{\colwidth}
\setlength{\sepwidth}{0.0333\paperwidth}
\setlength{\colwidth}{0.45\paperwidth}
\newcommand{\separatorcolumn}{\begin{column}{\sepwidth}\end{column}}
%\setlength\abovecaptionskip{-3pt}

% ====================
% Title
% ====================
\title{Observational Constraints on Models of Energy Release in Long-lived Active Region Loops}
\author{
  W. T. Barnes \inst{1}\textsuperscript{,}\inst{2} \and
  H. P. Warren \inst{3} \and
  J. W. Reep \inst{3}
}
\institute[]{
  \inst{1} Department of Physics, American University \samelineand
  \inst{2} Heliophysics Science Division, NASA Goddard Space Flight Center \and
  \inst{3} Space Science Division, Naval Research Laboratory
}

% ====================
% Footer (optional)
% ====================
\footercontent{
  \href{https://github.com/wtbarnes/hinode-2023-loops-poster}{github.com/wtbarnes/hinode-2023-loops-poster} \hfill
  Hinode-16/IRIS-13 --- Niigata, Japan --- 25--29 September 2023 \hfill
  \href{mailto:wbarnes@american.edu}{wbarnes@american.edu}
}

% ====================
% Logo (optional)
% ====================
\logoright{\includegraphics[height=8cm]{static/AU_stacked_logo.png}}
\logoleft{\includegraphics[height=10cm]{static/sunpy_logo_portrait_powered.png}}

% ====================
% Body
% ====================

\begin{document}

\begin{frame}[t]
\begin{columns}[t]
\separatorcolumn

\begin{column}{\colwidth}

  \begin{block}{Introduction}

    \begin{itemize}
      \item Long "warm" EUV loops on periphery of active regions (ARs) observed to:
      \begin{itemize}
        \item Be \alert{approximately steady}, lasting much longer than a cooling time
        \item Have \alert{narrow, nearly isothermal emission measure distributions centered near \SI{1}{\mega\kelvin}} \citep{aschwanden_three-dimensional_1999,aschwanden_evidence_2000}
        \item Be \alert{overdense} compared to densities expected from hydrostatic scaling laws \citep{winebarger_transition_2003}
        \item Have \alert{flat temperature structures} as determined by the \SI{195}{\angstrom}/\SI{171}{\angstrom} filter ratio \citep{aschwanden_three-dimensional_1999,lenz_temperature_1999}
      \end{itemize}
      \item \textbf{\alert{Problem:}} hydrostatic models fail to produce sufficiently high densities while impulsive heating scenarios lead to overly broad temperature distributions
      \item \textbf{\alert{Goal:}} Constrain observational properties of long warm loops spatially, temporally and over a broad range of temperatures and wavelengths
      \item \textbf{\alert{Goal:}} Constrain possible time-independent heating scenarios by comparing forward-modeled observables from field-aligned hydrodynamic models to observational diagnostics
    \end{itemize}

  \end{block}

  \begin{block}{NOAA AR 1575 Oberved by AIA, EIS, and XRT}

    \begin{figure}
      \centering
      \import{figures/}{ar_maps.pgf}
      \caption{NOAA AR 1575 as observed on 2012-09-24 by SDO/AIA, \textit{Hinode}/EIS, and \textit{Hinode}/XRT. The red dots and black lines denote the manually-identified loop. Each observation is cropped to the field of view of the EIS raster.}
      \label{fig:ar_maps}
    \end{figure}

    \begin{itemize}
      \item EIS raster started 2012-09-24 10:50:26 and ended 2012-09-24 13:56:09
      \item XRT synoptic data available for 2012-09-24 10:03
      \item AIA data selected between 2012-09-24 09:00:00 and 2012-09-24 21:00:00
      \item Fit 38 lines in each pixel of EIS raster to derive intensity and velocity using \texttt{eispac} \citep{weberg_eispac_2023}
      \item Align AIA EUV images to EIS FOV, normalizing by exposure time and correcting for degradation using \texttt{aiapy} \citep{barnes_aiapy_2020} 
    \end{itemize}

  \end{block}

  \begin{block}{Isolating Loop Structures}
    \begin{itemize}
      \item Manually identify points in time-averaged Gaussian-filtered \SI{171}{\angstrom} images that fall within EIS FOV and persist for $\approx\SI{12}{\hour}$.
      \item Create 2D pixel mask from each image assuming a loop width of \SI{45}{\arcsecond} with dimensions corresponding to the loop-aligned ($s_\parallel$) and cross-loop ($s_\perp$) directions.
      \item To further isolate loop structure, compute linear background model from emission at $s_\perp=\SI{0}{\arcsecond},\SI{45}{\arcsecond}$
    \end{itemize}
    \begin{figure}[H]
      \centering
      \import{figures/}{straightened_loops.pgf}
      \caption{Background-subtracted ``straightened'' loops as a function of $s_\parallel$ and $s_\perp$ as extracted from each image in Fig. \ref{fig:ar_maps}. Pixels are masked white if they fall below the background.}
      \label{fig:straightened_loops}
    \end{figure}

  \end{block}
  \vspace{-10px}
  \begin{block}{Density Diagnostics}

    \begin{itemize}
      \item Use method of \citet{young_high-precision_2009} to derive density from Fe XII \SI{186.880/195.119}{\angstrom} and Fe XIII \SI{203.826/202.044}{\angstrom} line pairs as observed by EIS
      \item Density structure approximately flat as a function of $s_\parallel$, both \alert{density distributions peaked at $\approx\SI{1e9}{\per\cm\cubed}$}
    \end{itemize}
    \vspace{-50px}
    \begin{figure}
      \centering
      \import{figures/}{density_diagnostics.pgf}
      \caption{Derived density from the Fe XII (top row) and Fe XIII (second row) line ratios. \textbf{Bottom left:} Theoretical line ratio as a function of density for the Fe XII and XIII line pairs. \textbf{Bottom right:} Distribution of densities from the maps in the first two rows.}
      \label{fig:density_diagnostic}   
    \end{figure}

  \end{block}
  \vspace{-30pt}
  \begin{block}{Temperature Diagnostics}

    \vspace{-30pt}

    \begin{figure}[H]
      \centering
      \import{figures/}{temperature_diagnostics.pgf}
      \caption{Temperature as derived from the ratio of the \SI{193}{\angstrom} and \SI{171}{\angstrom} filters. \textbf{Top:} Time-averaged temperature as a function of $s_\parallel,s_\perp$. \textbf{Middle:} Temperature averaged over $s_\perp$ as a function of the \SI{12}{\hour} AIA observing window. \textbf{Bottom:} Time- and $s_\perp$-averaged temperature. The shaded region denotes $1\sigma$ of the distribution of temperatures in time and $s_\perp$.}
      \label{fig:filter_ratio}
    \end{figure}

    \begin{figure}[H]
      \centering
      \import{figures/}{dem.pgf}
      \caption{\textbf{Left:} Contribution functions for all observed EIS lines and temperature response functions for several filters from AIA and XRT. \textbf{Right:} EM \textit{loci} curves and inverted emission measure distribution (black) computed using the MCMC method of \citet{kashyap_markov-chain_1998}. The intensities are averaged in $s_\parallel$ and the value of $s_\perp$ is chosen where the Fe XII \SI{195.119}{\angstrom} intensity is maximized. For AIA, the timestep closest to the middle of the EIS raster is selected.}
      \label{fig:dem_observed}
    \end{figure}

  \end{block}

\end{column}

\separatorcolumn

\begin{column}{\colwidth}

  \begin{block}{Time Lag Analysis}

    \begin{itemize}
      \item Apply time-lag analysis method \citep{viall_evidence_2012,barnes_understanding_2019} to understand cooling patterns along loop.
      \item Spatially coherent positive time lags in the 193-171 and 211-193 channel pairs indicate cooling between \alert{\SI{2}{\mega\K} to just below \SI{1}{\mega\K}}
      \item Incoherent time lags in 335-211 pair indicate \alert{little coherent evolution above \SI{2}{\mega\K}}
      \item Zero time lags in 171-131 pair indicate \alert{no cooling below \SI{0.5}{\mega\K}}
    \end{itemize}
    \vspace{-50pt}
    \begin{figure}
      \centering
      \import{figures/}{time_lags.pgf}
      \caption{Time lag analysis of the non-background-subtracted traced loops for four different AIA channel pairs. The value of each pixel denotes the time lag, in seconds, that maximizes the cross-correlation between the two channels. This analysis suggests that \alert{the temperature evolution of the loops is primarily int the \SIrange{1}{2}{\mega\kelvin} range.}}
      \label{fig:timelags}
    \end{figure}
  
  \end{block}

  \begin{block}{Constraining Loop Geometries}

    \vspace{-30pt}

    \begin{figure}
      \centering
      \import{figures/}{aia_fieldlines_b_profiles.pgf}
      \caption{\textbf{Left:} AIA \SI{171}{\angstrom} observation with field lines from a PFSS model overlaid. Field lines are selected by finding those that intersect with the polygon defined by the observed loop structure. \textbf{Right:} Magnetic field strength as a function of $s$ for each model field line. The blue lines denote $B(s)$ for each field line and the orange line denotes the mean magnetic field profile, $\bar{B}(s)$. The dashed orange line denotes the area expansion profile as calculated by $\bar{B}_{max}/\bar{B}(s)$.}
      \label{fig:aia_fieldlines_overlaid}
    \end{figure}
    
  \end{block}

  \begin{block}{Hydrodynamic Modeling}

    \begin{columns}[c]
      \begin{column}{0.6\colwidth}
        \begin{figure}
          \centering
          \import{figures/}{hydrodynamic_results.pgf}
          \caption{Temperature (solid, left axis) and density (dashed, right axis) as a function of field-aligned coordinate $s$ for the heating scenarios in Table \ref{tab:heating_scenarios}. $T_e$ and $n$ are time-averaged over a \SI{12}{\hour} interval. Except for run 3, there is minimal variation in $T_e$ and $n$ with $t$.}
          \label{fig:temperature_density_profiles}
        \end{figure}
      \end{column}
      \begin{column}{0.4\colwidth}
        \small
        \begin{table}
          \begin{tabular}{ccccc}
            \toprule
            Run & $s_H$ & $E_0$ & $f_{LR}$ & Expansion? \\
                & [Mm] & [erg cm$^{-3}$ s$^{-1}$] &  &  \\
            \midrule
            1 & $\infty$ & 8.7e-06 & 1 & No \\
            2 & $\infty$ & 8.7e-06 & 1 & Yes \\
            3 & 3.53 & 0.02 & 0.5 & Yes \\
            4 & 3.53 & 0.02 & 0 & No \\
            5 & 3.53 & 0.02 & 0.25 & No \\
            \bottomrule
          \end{tabular}
          \label{tab:heating_scenarios}
          \caption{Heating parameters for each run of the HYDRAD model \citep{bradshaw_self-consistent_2003,bradshaw_influence_2013}. In HYDRAD, volumetric heating can be parameterized as a function of location $s_0$ and scale height $s_H$, $E = E_0\exp{\left[-(s-s_{i,0})^2/2s_H^2\right]}$. Energy is deposited at each leg of the loop, with $f_{LR}$ being the balance of heating between the left (0) and right (1) footpoints. For runs 3-5, the heat is deposited at a height of $s_0=\SI{24.5}{\mega\meter}$ above the base of the loop. In each run, the gravitational stratification is determined from the mean gravitational profile of all extrapolated loops in Fig. \ref{fig:aia_fieldlines_overlaid} and the input loop length is \SI{353}{\mega\meter}, the average loop length. The expansion used in runs 2 and 3 is shown in the right panel of Fig. \ref{fig:aia_fieldlines_overlaid}.}
        \end{table}          
      \end{column}
    \end{columns}
  \end{block}

  \begin{block}{Synthetic Observables}

    \begin{figure}
      \centering
      \import{figures/}{synthetic_emission.pgf}
      \label{fig:synthetic_emission}
      \caption{Forward-modeled emission for 2 of the 5 possible heating scenarios for AIA \SI{171}{\angstrom} and \SI{193}{\angstrom}, EIS \SI{195.119}{\angstrom} and XRT Al-poly. To create the synthetic maps, a HYDRAD run is mapped to the spatial structure of the extrapolated field line and the emissivity is summed along the LOS using the \texttt{synthesizAR} Python package \citep{barnes_understanding_2019}.}
    \end{figure}

    \vspace{-50pt}

    \begin{figure}
      \centering
      \import{figures/}{synthetic_temperature_density_diagnostics.pgf}
      \label{fig:synethetic_diagnostics}
      \caption{\textbf{Left:} Density measurements as computed from forward-modeled intensities for each heating scenario using the method of \citet{young_high-precision_2009}. The solid lines denote the densities derived from the Fe XII intensities and the dashed lines denote the densities derived from Fe XIII. \alert{All models are unable to account for observed densities > \SI{1e9}{\per\cubic\cm}}. \textbf{Right:} Temperatures derived from the ratio of the forward-modeled AIA \SI{193}{\angstrom} and \SI{171}{\angstrom} intensities.\alert{Temperatures from the forward-modeled intensities are much more broadly distributed, indicating a less flat temperature profile compared to the observed temperature profiles.}}
    \end{figure}

  \end{block}

  \vspace{-10pt}

  \begin{block}{Assessing Heating Models}

    \begin{columns}[c]
      \begin{column}{0.4\colwidth}
        \begin{table}
          \centering
          \begin{tabular}{c c c c}
            \toprule
            \textbf{Run} & \textbf{Overdense?} & \textbf{Isothermal?} & \textbf{Flat?} \\
            \midrule
            1 & No & Yes &  No \\
            2 & No & Yes & Maybe? \\
            3 & No & No &  Maybe? \\
            4 & Maybe? & Yes &  No \\
            5 & Maybe? & Yes & No \\
            \bottomrule
          \end{tabular}
        \end{table}    
      \end{column}
      \begin{column}{0.59\colwidth}
        \begin{itemize}
          \item All models show bulk of densities below \SI{1e9}{\per\cubic\cm}
          \item Runs 4 and 5 (siphon flow) most promising, but \alert{velocities too high,$T$ not flat}
          \item Time-dependent solutions of $L\sim\SI{300}{\mega\m}$ loops are sufficiently overdense, but too broad in $T$
          \item \alert{None of the heating scenarios can account for all observable constraints}--missing physics in models?
        \end{itemize}
      \end{column}
    \end{columns}

  \end{block}

  \begin{block}{References}
    \tiny
    WTB was supported by NASA's \textit{Hinode} program.
    \textit{Hinode} is a Japanese mission developed and launched by ISAS/JAXA with NAOJ as a domestic partner and NASA and STFC (UK) as international partners.
    It is operated by these agencies in cooperation with ESA and NSC (Norway).
    \begin{multicols}{2}
      \bibliographystyle{aasjournal.bst}
      \bibliography{references.bib}
    \end{multicols}
  \end{block}

\end{column}

\separatorcolumn
\end{columns}
\end{frame}

\end{document}
